\section{Introdução}
\label{introduction}

A qualidade interna é um dos fatores de sucesso de projetos de software, pois corresponde a aspectos primordiais do software tais como manutenibilidade e segurança. Softwares com boa qualidade interna proporcionam maior produtividade uma vez que possibilitam a criação de mais testes automatizados, são mais compreensíveis, reduzem o risco de \emph{bugs} e facilitam as modificações e evoluções no código. Além disso, um estudo do ICAT/NIST\footnote{ICAT foi um motor de busca de vulnerabilidades, desenvolvido pelo NIST(National Institue of Standards and Technology), catalogadas no padrão \emph{Common Vulnerabilities and Exposures} - CVE. O ICAT foi substituido pelo NVD (National Vulnerability Database) que, além de possuir o mesmo mecanismo de busca, é um repositório governamental dos Estados Unidos que armazena diversas informações sobre vulnerabilidades de software (nomenclaturas, métricas, checklists, etc).} de 2005 já apontava que 80\% das vulnerabilidades exploráveis estão ligadas a má codificação.

Portanto, o Engenheiro de Software é um dos responsáveis por esse sucesso, uma vez que deve reunir um conjunto de habilidades e conhecimentos que o permitam aplicar práticas, técnicas e ferramentas para a criação de softwares seguros e com bom \emph{design}. Nesse sentido, a medição pode ser utilizada como um processo de apoio ao acompanhamento da segurança e qualidade, através do estabelecimento de metas e indicadores que indiquem oportunidades de melhorias observáveis do produto. Em um cenário otimista,
os próprios Engenheiros de Software podem adotar como prática a medição do código-fonte para auxiliar as tomadas de decisões, ou até mesmo para avaliação do código inserido ou da aplicação de refatorações. Entretanto, uma grande quantidade de métricas, coletas manuais e poucos recursos de visualização são fatores que acabam por desmotivar o uso dessas para o monitoramento do código. Além disso, assim como destacado por \cite{chidamber1994}, a compreensão do significado de valores obtidos através de métricas não é uma tarefa trivial, demandando um grande esforço de coleta e interpretação, cujas análises podem ser errôneas caso feitas sobre métricas isoladas, correlações inexistes ou até mesmo a escolha de métricas inadequadas.

Métricas são ferramentas importantes para o Engenheiro de Software, pois podem ser utilizadas na Engenharia de Software Experimental \cite{hegedus2012} e em projetos de desenvolvimento de software. Mais especificamente, métricas de código-fonte podem ser utilizadas tanto no âmbito gerencial quanto como referência técnica para tomada de decisões sobre o código-fonte. Métricas de código-fonte possuem natureza objetiva que buscam mensurar tamanho, complexidade e outros atributos importantes do software \cite{henry1984kafura}\cite{troy1981zweben}\cite{yau1985zweben}\cite{systa2000}. Neste sentido, o estudo de métricas e sua utilização no contexto de segurança e qualidade interna do código-fonte podem ser fundamentais para a formação do Engenheiro de Software.

Dada a importância do tema para a formação profissional do Engenheiro de Software, neste artigo iremos mostrar os estudos e resultados intermediários de um trabalho de conclusão de curso de Engenharia de Software da Faculdade UnB Gama cujo foco é explorar e potencializar a utilização de métricas para o monitoramento de código-fonte a partir da compreensão e estabelecimento de possíveis relações existentes entre métricas de vulnerabilidade e qualidade de software. Assim, esse trabalho contempla a composição de métricas em cenários de decisões e o desenvolvimento e evolução de duas ferramentas que possam apoiar a utilização de métricas e cenários na melhoria contínua do desenvolvimento. A primeira ferramenta consiste em uma plataforma livre de monitoramento de código fonte chamada Mezuro, enquanto a segunda consiste em um ambiente de \emph{Data Warehousing} que também já foi explorado em outros trabalhos \cite{Folleco2007}\cite{Silveira2010}\cite{mazuco2011}.

No presente texto iremos apresentar na Seção \ref{sec:methodology} como foi nossa organização e metodologia utilizada durante a execução do trabalho de conclusão de curso. Posteriormente, a Seção \ref{sec:studies} apresenta os estudos e resultados obtidos até o momento no trabalho. Por fim, a Seção \ref{sec:final-remarkds} conclúi este artigo e mostra os trabalhos futuros.
