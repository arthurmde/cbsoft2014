\section{Estudos e Resultados}
\label{sec:studies}

Dentro do trabalho de conclusão de curso, destinamos um capítulo completo para estudos relacionados ao \emph{design} e segurança de software cujo principal objetivo era compreender os principais aspectos, problemas, princípios, técnicas e práticas as quais o Engenheiro de Software está exposto a partir de uma revisão bibliográfica com mais de 50 fontes. Da mesma forma, dedicamos uma capítulo para explorar os principais conceitos relacionados a métricas estáticas de código-fonte, onde também listamos um conjunto de métricas de \emph{design} de software e um conjunto de métricas de vulnerabilidades.

%A segurança de software está relacionada com o contínuo processo de manter a confidencialidade, integridade e disponibilidade nas diversas camadas que o compõe. Vulnerabilidades podem ser exploradas de maneira maliciosa para permitir acesso não autorizado, modificações de privilégios e negação de serviço. McGraw e colaboradores \cite{mcgraw2004} afirmam que 50\% dos problemas de segurança surgem no nível de \emph{design}. Dessa forma, assim como os desenvolvedores programam aplicando ao código princípios de \emph{design}, devem evoluir o código aplicando princípios de \emph{design} seguro tais quais os apresentados por outros trabalhos \cite{saltzer1975}. E no meio de uma gama de vulnerabilidades existentes, iniciativas como a CWE\footnote{\url{http://cwe.mitre.org/about/sources.html}} provêm uma base de informações padrão que ajudam os Engenheiros de Software a identificar, mitigar e previnir uma certa vulnerabilidade.

Com os estudos verificamos uma forte relação entre princípios de \emph{design} de software com os princípios de \emph{design} seguro, onde a aplicação de ambos podem prover softwares mais robustos, extensíveis e seguros. Os cuidados com o bom \emph{design} do código-fonte são fundamentais para o desenvolvimento de códigos seguros, podendo ser realizados através, por exemplo, da prática de \emph{refactorings} e aplicação de padrões de projeto. Neste sentido, o monitoramento do código fonte pode ser utilizado para apoiar a utillização de técnicas que buscam aplicar os princípios citados e este monitoramento pode ser feito com o auxílio de métricas de código fonte.

Com o objetivo de minimizar as principais dificuldades existentes na medição de código fonte, como a escolha de métricas, interpretação de valores, redundância de métricas e interpretaçoes isoladas, buscamos definir o conceito de Cenários de Decisão. Os Cenários de Decisão  visam nomear e mapear estados observáveis através de métricas de código-fonte que indicam a existência de determinada característica dentro do software, classe ou método. A estrutura dos cenários consistem em:

\begin{itemize}
\item \textbf{Nome}: Identificação única do cenário
\item \textbf{Métricas Envolvidas}: Métricas necessárias para a caracterização do cenário
\item \textbf{Nível}: Abstração envolvida (projeto, classe, método)
\item \textbf{Descrição}: Discuti os problemas, princípios envolvidos e a caracterização
\item \textbf{Caracterização com Métricas}: Define e discuti como as métricas envolvidas devem ser utilizadas para identificar o cenário
\item \textbf{Ações Sugeridas}: Propõe um conjunto de ações específicas tais como uma refatoração, a utilização de um padrão de projeto, prática e aplicação de princípios
\end{itemize}

A proposta dessa estrutura de cenários de decisões tem como principal objetivo abstrair a escolha e interpretações de métricas de código-fonte, reduzindo erros e potencializando o uso de métricas para tomada de decisões em projetos. Neste sentido, projetos podem utilizar cenários de referência ou até mesmo definir novos cenários de acordo com parâmetros de qualidade do projeto. Dentro da primeira etapa de desenvolvimento desse trabalho, baseado nos estudos e relações existentes sobre \emph{design} e vulnerabilidades de software, definimos um conjunto de cenários que visam identificar características de software relacionado à violações de princípios de segurança e presença de vulnerabilidades.

Por fim, na primeira etapa dessa monografia, contribuímos tecnicamente com dois projetos de software livre que fazem parte do contexto deste trabalho: o Mezuro\footnote{\url{http://mezuro.org/}} e Analizo\footnote{\url{http://analizo.org/}}, que é um dos extratores utilizados pelo Mezuro. Os objetivos dessas contribuições foi compreender a arquitetura desses projetos para proporcionar evoluções que contemplem a utilização de cenários e a coleta e apresentação de novas métricas de vulnerabilidades.
