\section{Metodologia}
\label{sec:methodology}

Para identificar a relação entre métricas de design com métricas de vulnerabilidade, buscamos primeiramente  realizar um estudo teórico por meio de uma revisão bibliográfica sobre ambos os aspectos. Este estudo nos deu insumo para correlacionar teoricamente algumas características de vulnerabilidades e de desing, além de correlacionar algumas métricas, que nos permitiu a criação de cenários de decisão originidos a partir dessa correlação. Apoiados por algumas iniciativas da utilização de métricas compostas em cenários, como feito em \ref{almeida2010}, buscamos definir melhor e protocoloar (MUDAR ESSE TERMO???) o cenário de decisão e propor alguns cenários para o contexto abordado na monografia.

O desenvolvimento da monografia em dupla exigiu que tomássemos algumas medidas a fim de organizar oque estava sendo produzido para que conseguissemos gerar um trabalho consistente e coerente. A partir de experiênncias com métodos ágeis e desenvolvimento de Software livre buscamos montar algumas estratégias para gerenciar o desenvolvimento do trabalho. Primeiramente, foi essencial o uso de uma ferramenta de controle de versão,  pois deixa todo o conteúdo armazenado em um repositório central acessível a todos e permite ambos trabalhem no texto com menos problemas de conflitos. Para esse propósito foi utilizado o GitHub como repositório. Buscamos trabalhar diáriamente, realizando alguns encontros para discutir assuntos pertinentes e definir alguns aspectos. Utilizamos "Issues" para reportar os problemas que deveriam ser solucionados, assim como é feito no desenvolvimento de softwares livres, assim, os nossos objetivos eram solucionar todas as issues.   
