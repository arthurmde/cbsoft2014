\section{Metodologia}
\label{sec:methodology}

Este trabalho tem sido realizado por dois graduandos de Engenharia de Software cuja primeira etapa foi desenvolvida ao longo do 1º semestre de 2014 e a finalização deve-se dar no fim do 2º semestre do mesmo ano.
%

O desenvolvimento da monografia em dupla exigiu que tomássemos algumas medidas a fim de organizar o que estava sendo produzido para que fosse possível gerar um trabalho consistente e coerente. A partir de experiências com métodos ágeis e desenvolvimento de Software livre buscamos montar algumas estratégias para gerenciar o desenvolvimento do trabalho. Primeiramente, foi essencial o uso de uma ferramenta de controle de versão para controle do texto produzido em Latex e a utilização de um repositório central acessível a todos, permitindo que ambos trabalhem no texto com menos problemas de conflitos. Para esse propósito foi utilizado o Git como ferramenta de controle de versão e o GitHub como repositório central. Buscamos trabalhar diariamente sobre o texto e pesquisas, realizando alguns encontros para discutir assuntos pertinentes e definir alguns aspectos. As principais decisões a respeito do tema do trabalho e propostas de abordagens para alcançar os objetivos do mesmo foram tomadas a partir de reuniões semanais, onde planejamos um conjunto de "\emph{issues}" relacionados à realização do trabalho técnico e escrita do texto. "\emph{Issues}" são ferramentas oferecidas pelo GitHub e muito utilizadas por comunidades de softwares livres para gerenciar o desenvolvimento de projetos, correção de problemas. Neste sentido, semanalmente tinhámos um conunto de "\emph{issues}" que precisávamos fechar para alançar os objetivos planejados.
