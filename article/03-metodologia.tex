\section{Metodologia}
\label{sec:methodology}

Este trabalho tem sido realizado por dois graduandos de Engenharia de Software cuja primeira etapa foi desenvolvida ao longo do 1º semestre de 2014 e a finalização deve-se dar no fim do 2º semestre do mesmo ano.
%

Dentro da primeira etapa, realizamos uma extensa revisão bibliográfica com mais de 50 fontes de estudos relacionados ao \emph{design} e segurança de software cujo principal objetivo era compreender os principais aspectos, problemas, princípios, técnicas e práticas as quais o Engenheiro de Software está exposto. Da mesma forma, exploramos os principais conceitos relacionados a métricas estáticas de código-fonte, onde também listamos um conjunto de métricas de \emph{design} de software e um conjunto de métricas de vulnerabilidades. As métricas foram explanadas foram selecionadas devido sua popularidade tanto em trabalhos científicos quanto em ferramentas de análise estática de código-fonte.

A definição de estrutra dos Cenários de Decisões, explicados na próxima sessão, foi advinda do estudo realizado por Almeira \& Miranda (\citeyear{almeida2010}) sobre mapeamento de métricas de código-fonte com os conceitos de Código Limpo. A partir dos estudos realizados sobre métricas, \emph{design} e segurança, foram propostos alguns cenários para tomada de decisões sobre projetos de software, principalmente sobre segurança e vulnerabilidade.