\section{Metodologia}
\label{sec:methodology}

O presente texto apresenta os resultados obtidos durante a primeira etapa de um trabalho de conslusão de curso cujos principais objetos de estudos são métricas de código-fonte. Este trabalho tem sido realizado por dois graduandos de Engenharia de Software da Faculdade UnB Gama através da orientação de dois professores da mesma unidade acadêmica e está alinhado com os principais objetivos de pesquisa do Laboratório Avançado de Produção Pesquisa e Inovação em Software - LAPPIS. 

O desenvolvimento da monografia em dupla exigiu que tomássemos algumas medidas a fim de organizar o que estava sendo produzido para que fosse possível gerar um trabalho consistente e coerente. A partir de experiências com métodos ágeis e desenvolvimento de Software livre buscamos montar algumas estratégias para gerenciar o desenvolvimento do trabalho. Primeiramente, foi essencial o uso de uma ferramenta de controle de versão para controle do texto produzido em Latex e a utilização de um repositório central acessível a todos, permitindo que ambos trabalhem no texto com menos problemas de conflitos. Para esse propósito foi utilizado o Git como ferramenta de controle de versão e o GitHub como repositório central. Buscamos trabalhar diariamente sobre o texto e pesquisas, realizando alguns encontros para discutir assuntos pertinentes e definir alguns aspectos. As principais decisões a respeito do tema do trabalho e propostas de abordagens para alcançar os objetivos do mesmo foram tomadas a partir de reuniões semanais, onde planejamos um conjunto de "\emph{issues}" relacionados à realização do trabalho técnico e escrita do texto. "\emph{Issues}" são ferramentas oferecidas pelo GitHub e muito utilizadas por comunidades de softwares livres para gerenciar o desenvolvimento de projetos, correção de problemas. Neste sentido, semanalmente tinámos um conunto de "\emph{issues}" que precisávamos fechar para alançar os objetivos planejados.

Dentro da monografia, destinamos um capítulo completo para estudos relacionados ao \emph{design} e segurança de software cujo principal objetivo era compreender os principais aspectos, problemas, princípios, técnicas e práticas as quais o Engenheiro de Software está exposto a partir de uma revisão bibliográfica com mais de 50 fontes. Da mesma forma, dedicamos uma capítulo para explorar os principais conceitos relacionados a métricas estáticas de código-fonte, onde também listamos um conjunto de métricas de \emph{design} de software e um conjunto de métricas de vulnerabilidades. Este estudo nos deu insumo para relacionar teoricamente algumas características de vulnerabilidades e de \emph{design} tanto a nível de princípios e práticas quanto a nível de métricas.

Apoiados por algumas iniciativas da utilização de métricas compostas em cenários, como feito em \ref{almeida2010}, buscamos compor as métricas em cenários de decisão para o contexto abordado na monografia. Para tanto, foi proposta uma estrutura de composição de cenários que visam nomear e mapear estados observáveis através de métricas de código-fonte que indicam a existência de determinada característica dentro do software, classe ou método. A estrutura dos cenários consistem em:
\begin{itemize}
\item \textbf{Nome}: Identificação única do cenário
\item \textbf{Métricas Envolvidas}: Métricas necessárias para a caracterização do cenário
\item \textbf{Nível}: Abstração envolvida (projeto, classe, método)
\item \textbf{Descrição}: Discuti os problemas, princípios envolvidos e a caracterização
\item \textbf{Caracterização com Métricas}: Define e discuti como as métricas envolvidas devem ser utilizadas para identificar o cenário
\item \textbf{Ações Sugeridas}: Propõe um conjunto de ações específicas tais como uma refatoração, a utilização de um padrão de projeto, prática e aplicação de princípios
\end{itemize}

Por fim, na primeira etapa dessa monografia, contribuímos tecnicamente com dois projetos de software livre que fazem parte do contexto deste trabalho: o Mezuro\footnote{\url{http://mezuro.org/}} e Analizo\footnote{\url{http://analizo.org/}}. Os objetivos dessas contribuições foi compreender a arquitetura desses projetos para proporcionar evoluções que contemplem a utilização de cenários e a coleta e apresentação de novas métricas de vulnerabilidades.

   
