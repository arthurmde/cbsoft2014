\documentclass[conference]{IEEEtran}
\usepackage[T1]{fontenc}	
\usepackage[utf8]{inputenc}

% correct bad hyphenation here
\hyphenation{op-tical net-works semi-conduc-tor}

\begin{document}
%
% paper title
% can use linebreaks \\ within to get better formatting as desired
\title{Tomadas de Decisão Orientado a Métricas de Software: Experências e Aprendizados ao trabalhar com métricas de Vulnerabilidade e de Qualidade Interna para melhoria de design de Software}


% author names and affiliations
% use a multiple column layout for up to three different
% affiliations
\author{\IEEEauthorblockN{Carlos Filipe Lima Bezerra}
\IEEEauthorblockA{Engenharia de Software\\Universidade de Brasília, DF\\
Email: carlosfilipe.lb@gmail.com}
\and
\IEEEauthorblockN{Arthur Del Esposte}
\IEEEauthorblockA{Twentieth Century Fox\\
Springfield, USA\\
Email: homer@thesimpsons.com}}



% make the title area
\maketitle


\begin{abstract}
%\boldmath
The abstract goes here.
\end{abstract}

\IEEEpeerreviewmaketitle



\section{Introdução}

A qualidade interna é um dos fatores de sucesso de projetos de software, pois corresponde a aspectos primordiais do software tais como manutenibilidade e segurança. Softwares com boa qualidade interna proporcionam maior produtividade uma vez que possibilitam a criação de mais testes automatizados, são mais compreensíveis, reduzem o risco de bugs e facilitam as modificações e evoluções no código. Além disso, um estudo do ICAT/NIST de 2005 já apontava que 80\% das vulnerabilidades exploráveis estão ligadas a má codificação.

Portanto, o Engenheiro de Software é um dos responsáveis por esse sucesso, uma vez que deve reunir um conjunto de habilidades e conhecimentos que o permitam aplicar práticas, técnicas e ferramentas para a criação de softwares seguros e com bom design. Nesse sentido,a medição pode ser utilizada como um processo de apoio ao acompanhamento da segurança e qualidade, através do estabelecimento de metas e indicadores que indiquem oportunidades de melhorias observáveis do produto. Em um cenário otimista,
os próprios Engenheiros de Software podem adotar como prática a medição do código-fonte para auxiliar as tomadas de decisões, ou até mesmo para avaliação do código inserido ou da aplicação de refatorações. Entretanto, uma grande quantidade de métricas, coletas manuais e poucos recursos de visualização são fatores que acabam por desmotivar o uso dessas para o monitoramento do código. Além disso, a compreenção do significado de valores obtidos através de métricas não é uma tarefa trivial, demandando um grande esforço de interpretação necessárias para a tomada de decisão efetiva sobre o projeto de software.

Portanto, neste trabalho serão exploradas a utilização de métricas para o monitoramento de código-fonte para compreender e estabelecer possíveis relações existentes entre as mesmas no que diz respeito a vunerabilidades e qualidade de software. A partir do estabelecimento de cenários esperamos identificar oportunidades de utilização de métricas na melhoria contínua do desenvolvimento e, consequentemente, na qualidade interna do produto. Com o objetivo de facilitar a interpretação e evitar possíveis equívocos, que são baseados em análises errôneas sobre métricas isoladas, correlações inexistes ou até mesmo a escolha de métricas inadequadas cujo problemas são discutidos em (CHIDAMBER; KEREMER, 1994), estes cenários serão compostos a partir da análise de relação e correlação entre métricas. Tal correlação buscará evidenciar boas e más características de bom design de um projeto que impactam em vulnerabilidades no sistema. Para auxiliar no monitoramento e na tomada de decisão, será explorado o uso de plataforma de monitoramento de código-fonte Mezuro e um ambiente de DWing. Dessa forma, será observado o uso dessas duas soluções e suas contribuições para a melhoria do desenvolvimento e qualidade do produto. Também mostraremos a metodologia que utilizamos para a criação deste trabalho, assim como os aspectos que são importantes para nossa formação ao arbordar este asssunto. (MELHORAR ESSE FINAL.. talvez encaixar em outro lugar... )

\hfill mds
 
\hfill January 11, 2007

\subsection{Subsection Heading Here}
Subsection text here.


\subsubsection{Subsubsection Heading Here}
Subsubsection text here.


\section{Conclusão}
The conclusion goes here.




% conference papers do not normally have an appendix


% use section* for acknowledgement
\section*{Acknowledgment}


The authors would like to thank...



% references section

% can use a bibliography generated by BibTeX as a .bbl file
% BibTeX documentation can be easily obtained at:
% http://www.ctan.org/tex-archive/biblio/bibtex/contrib/doc/
% The IEEEtran BibTeX style support page is at:
% http://www.michaelshell.org/tex/ieeetran/bibtex/
%\bibliographystyle{IEEEtran}
% argument is your BibTeX string definitions and bibliography database(s)
%\bibliography{IEEEabrv,../bib/paper}
%
% <OR> manually copy in the resultant .bbl file
% set second argument of \begin to the number of references
% (used to reserve space for the reference number labels box)
\begin{thebibliography}{1}

\bibitem{IEEEhowto:kopka}
H.~Kopka and P.~W. Daly, \emph{A Guide to \LaTeX}, 3rd~ed.\hskip 1em plus
  0.5em minus 0.4em\relax Harlow, England: Addison-Wesley, 1999.

\end{thebibliography}




% that's all folks
\end{document}


