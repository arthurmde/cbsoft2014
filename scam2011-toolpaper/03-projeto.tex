\section{Projeto}
\label{sec:projeto}


O Portal do Software Público Brasileiro, SPB, inaugurado em 2007, na prática, é um sistema
web que se consolidou como um ambiente de compartilhamento de projetos de software.
Oferece um espaço (comunidade) para cada software. A comunidade é composta por fórum,
notícias, chat, armazenamento de ae rquivos e downloads, wiki, lista de prestadores de
serviços, usuários, coordenadores, entre outros recursos. Teve um crescimento expressivo
contando, hoje, com mais de 60 comunidades de desenvolvimento e mais de 200.000 usuários
cadastrados. O SPB abrange também, o 4CMBr que é o grupo de interesse voltado para
soluções de tecnologia para municípios, o 5CQualiBr que é um grupo que trabalha para
evoluir a qualidade do Software Público Brasileiro, o 4CTecBr, um portal destinado a
colaboração no desenvolvimento de Tecnologias Livres, o Mercado Público Virtual que é um
grupo de empresas e pessoas que prestam serviço nos softwares ofertados no Portal e o Ava- 
liaSPB que avalia a entrada dos softwares candidatos a software público. O ambiente do 
SPB não proporciona a integração com ambientes colaborativos externos, especialmente com
redes sociais. A plataforma escolhida na ocasião da criação foi o framework OpenACS, que
continua sendo utilizada na atual versão.
%
A evolução do SPB foi comprometida desde 2009, quando framework OpenACS foi 
descontinuado. Com isso, não tendo versões lançadas a partir daquele ano. Por isso, hoje,
é necessária a evolução para novas tecnologias, que tenham maior suporte das comunidades
de desenvolvimento, utilize linguagens de programação com maior velocidade de
desenvolvimento e permita a integração com ambientes colaborativos externos, em especial,
redes sociais. Além disso, é preciso realizar a manutenção evolutiva das funcionalidades
existentes e também o desenvolvimento de novas funcionalidades para o Portal do SPB .
%
Um dos passos para a concretização de uma nova geração do Portal SPB é a integração de
novas tecnologias, desde uma plataforma colaborativa até sistemas de controle de versão e
monitoramento da qualidade do código-fonte, gerenciadas e apresentadas em uma plataforma
integrada no back-end e, em especial, no front-end para que os usuários e as comunidades
dos projetos tenham um conjunto de recursos para encontrarem os projetos, bem como
colaborarem em torno de um sofware público.
%
Mesmo com as limitações citadas, o Portal do Software Público Brasileiro teve em 2013 
mais de 600 mil visitantes únicos, com mais de 1 milhão de visitas/acessos, gerando mais
de 16 milhões de visitadas nas páginas, com um total de mais de 49 milhões de hits no
Portal SPB. Avaliando apenas as comunidades dos projetos I3Geo, CAU, CACIC e Geplanes,
houve mais de 15 mil downloads e 4 mil mensagens nos fórum. Essa amostra estatística
ilustra bem o potencial do Software Público Brasileiro, bem como as expectativas de seus
usuários e colaboradores para a evolução do Portal e do modelo em si.