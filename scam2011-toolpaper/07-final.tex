\section{Final remarks}
\label{sec:final-remarkds}

This paper presented Kalibro Metrics, a representative of a new generation of
source code metrics analysis tool.
%
Currently, it is integrated with the Analizo metrics tool, supporting the
analysis of software projects written in the C, C++, and Java programming languages.
% 
Kalibro Metrics has useful features for both software engineering researchers
working with source code analysis and professionals who want to analyze their
source code to identify potential problems or possible enhancements.

Kalibro Metrics is fully flexible and allows easy integration with distinct
source code analysis tools.
%
Also, Kalibro provides an environment where software engineers
can define their own threshold configurations, according to software
implementation context and their experiences in software development.
%
These thresholds are shared among other software engineers through the Kalibro
Web Service.
%
Finally, each metric loaded within Kalibro can support as many thresholds as
possible to provide a full interpretation about what a metric value means.

Future works include the support of other kind of repositories such as
CVS and Bazaar.
%
In addition, the integration with other metric collector tools, especially
to provide Python and Ruby source code analysis.
%
Moreover, the development of Mezuro, a web-based social network environment
for source code tracking, analysis, and visualization, using Kalibro Plug-in
to connect to Kalibro Service.
%
Mezuro project is currently under development.

In conclusion, Kalibro Metrics is Free Software, licensed under the GNU Lesser
General Public License version 3. 
%
Its source code, as well as binary packages, manuals, tutorials,
and video demonstration can be found and obtained from \verb|http://kalibro.org|.

\section*{Acknowledgment}

The authors of this paper were supported by Brazilian Nation Research Council
(CNPq),  Coordination for the Improvement of Higher Level Personnel (CAPES),
and QualiPSo project.
%
The Kalibro Metrics tool has been developed as a USP FLOSS Competence
Center project, so the authors would like to thank Dr. Alfredo Goldman
for his collaborations through the eXtreme Programming Laboratory course.
%
Also, a special thanks to Dr. John Pearson for his review and
suggestions regarding this paper, as well as to thank his department to receive
PhD. student Paulo Meirelles as visiting research at Southern Illinois
University Carbondale (SIUC).
