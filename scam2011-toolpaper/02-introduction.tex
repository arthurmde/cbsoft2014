\section{Introduction}
\label{introduction}

From a practical point of view, whatever the methodology, software development
should be guided by two fundamental aspects to control the software quality
in the long term: source code and test quality.
%
Software testing does not mean to evaluate the source code quality. On the other
hand, software source code metrics can help software engineers to observe
the source code quality.
% 
Software Engineering requires the understanding of software, which is the
result from the writing of source code.
%
Also, software engineers and researchers need to analyze source codes to 
understand better software projects.

In the context of methodologies such as Agile Methods~\cite{beck1999} and ones
used by Free Software\footnote{In our work, we consider the terms
``free software'' and ``open source software'' equivalent.}~\cite{fsf}
communities, source code is the main artefact
of software development activities since features are constantly delivered to
customers and users.
%
Usually, software source code is written gradually and different
developers make updates and improvements on an ongoing basis~\cite{martin2008}.
%
%In the Free Software cases, only the successful ones.
%
Thus, new features are inserted and bugs are resolved during software
development and maintenance iterations.


An important observation on this process is that there is a significant gap
between the numbers of lines of code which a software engineer reads and writes.
%
Since developers read hundreds of lines of code (including their own past codes) to
understand an implementation to change it or write more code, source code should
be written to be read by other people~\cite{martin2008}.  


In addition, software engineers need to make decisions about their codes when
they are programming at the method and class level.
%
The sum of these decisions influences the source code quality of
their software~\cite{beck2007}.
%
To make the best decisions, they should monitor attributes from
their source codes. 
%
For example, in a collaborative software development such as Free Software,
source code organization and complexity influence the number of downloads and
members from a Free Software project \cite{meirelles2010}.   

In this scenario, source code metrics can support the development of clean code,
i.e., clear, flexible, and simple~\cite{martin2008}.
%
From an automated collection of metrics and an objective way to interpret their
values, software engineers can monitor specific characteristics of their code,
as well as detect troublesome scenarios to make decisions about the clarity,
flexibility and simplicity of their codes at the method and class level.

However, even with the fact that source code metrics have been proposed since the
1970s~\cite{SEI88}, there is not a set of standard measures established.
%
Moreover, little data are available or known to compare their effects,
as well as there is not a systematic approach to use, interpret, and
understand software metrics.
%
Thus, software measurement, specially the use of source code
metrics, are not fully explored in the software development~\cite{Tempero2008}.
%TODO: preciso colocar uma outra referencia aqui
%
Metric specialists are capable of using and understanding source code
metrics, but all software engineers should know how to use these metrics to
monitor and improve their source codes.

To make that easier, it is necessary for a tool to systematize the
source code evaluation.
%
We have identified, in particular, two limitations from current available tools,
i.e., lack of the following features:
%
(i) to collect automatically source code metrics values independent of the
programming language;  
(ii) to interpret measurement results, associating them with source code
quality.
%
The first is important because software engineers and developers should be able
to use the same tool for different languages, which would provide a similar set
of metrics.
%
The second takes a new approach how to analyze source code metrics values
since an interpretation of them out of the software implementation context can
not indicate and evaluate interesting source code attributes~\cite{SEI88}.


Since source code assessment tools frequently show their results as 
isolated numeric values for each metric, they are limited from the
perspective of software engineers who are not metric specialists. 
%
Therefore, in this paper, we present Kalibro Metrics, a Free Software tool
designed to incorporate any source code metric tool, extending it to provide
easy to understand evaluation of the analyzed software quality.
%
Kalibro Metrics allows a metric specialist user to specify a set of
acceptance ranges to each metric provided by its integrated source code metric
collector tools, as well as enabling the definition of customized metrics.
%
With a set of thresholds and their respective interpretations, any software
engineer can explore and better understand the use of source code metrics. 

The remainder of this paper is organized as follows:
%
%
Section \ref{sec:related-work} describes related work.
%
Section \ref{sec:architecture} describes Kalibro architecture.
%
Section \ref{sec:features} presents Kalibro features.
%
Section \ref{sec:use-cases} presents Kalibro use cases.
%
Finally, Section \ref{sec:final-remarkds} concludes the paper and discusses
future work.
